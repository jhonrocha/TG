%% LyX 2.2.2 created this file.  For more info, see http://www.lyx.org/.
%% Do not edit unless you really know what you are doing.
\documentclass[oneside,english,brazil,11pt,a4paper,openright,titlepage]{book}
\setcounter{secnumdepth}{3}
\setcounter{tocdepth}{3}
\usepackage{array}
\usepackage{graphicx, mathtools, bm}
\usepackage[utf8]{inputenc}

\makeatletter

%%%%%%%%%%%%%%%%%%%%%%%%%%%%%% LyX specific LaTeX commands.
%% Because html converters don't know tabularnewline
\providecommand{\tabularnewline}{\\}

%%%%%%%%%%%%%%%%%%%%%%%%%%%%%% User specified LaTeX commands.
% Classe alternativa, apropriada para impressão frente-verso. Inclui páginas em branco
% de forma que capítulos sempre tenham início na página à direita:
% \documentclass[11pt,a4paper,openright,titlepage]{book}

\usepackage[T1]{fontenc}
\usepackage{ft4unb_MKT}
\usepackage[portuguese]{babel}
\@ifundefined{showcaptionsetup}{}{%
 \PassOptionsToPackage{caption=false}{subfig}}
\usepackage{subfig}
\makeatother

\usepackage{babel}
\usepackage{indentfirst}

%%%%%%% My Includes %%%%%%%
\usepackage[hidelinks]{hyperref}
\usepackage{float}
\usepackage{amsfonts}
\usepackage[]{mcode}
\usepackage{epigraph}
\usepackage{pbox}

\newcommand{\jhhref}[2]{\hyperref[{#1}]{{#2} \ref{#1}}}
\newcommand{\jhmrow}[1]{\parbox[c][1.5cm][c]{3cm}{\centering #1}}
\newtheorem{mydef}{Definição}
\newtheorem{myteo}{Teorema}
\newtheorem{myexmp}{Exemplo}

\begin{document}
%CAPA
\grau{Engenheiro Mecatrônico}{Engenheiro} \tipodemonografia{Trabalho de Graduação}

% Título do trabalho (nao deletar linhas. Caso desejado, deixar em branco)
\titulolinhai{Implementação de Controle Fuzzy} 
\titulolinhaii{ Em CLP Industrial} 
\titulolinhaiii{} 
\titulolinhaiv{}
\tituloingles{Fuzzy Control Implementation in Industrial CLP}

%Autores
\autori{Jhonantans Moraes Rocha} 
\autorii{} % deixar em branco se não houver segundo autor. 
\autoriii{} % deixar em branco se não houver terceiro autor. 

\capaprincipal

%CONTRA CAPA

%Membros da banca (até 5 membros) 
\membrodabancai{Prof. Eduardo Stockler Tognetti, ENE/UnB} \membrodabancaifuncao{Orientador} 
\membrodabancaii{Prof. Eugênio Fortaleza, ENM/UnB} \membrodabancaiifuncao{Examinador Externo}
\membrodabancaiii{Dr. Luis Felipe Cruz Figueredo, ENE/UnB} \membrodabancaiiifuncao{Examinador Externo} 
\membrodabancaiv{} \membrodabancaivfuncao{} 
\membrodabancav{} \membrodabancavfuncao{}

%Data de defesa: dia, mês e ano. 
\dia{08} \mes{dezembro} \ano{2016}

\capaassinaturas

%FICHA CATALOGRAFICA

%Sobrenome, Nome Completo 
\autorcatalogo{Rocha, Jhonantans Moraes} 
%Sobrenome, Iniciais 
\autorabreviadocatalogo{Rocha, J.M.} 
%Número de páginas
\numeropaginascatalogo{69p.} 
\palavraschavecatalogoi{Fuzzy} 
\palavraschavecatalogoii{Controle} 
\palavraschavecatalogoiii{Quatro-Tanques} 
\palavraschavecatalogoiv{CLP} 
%Número da publicação (fornecido pelo departamento após a defesa) 
\publicacao{TG-037/2016}

\fichacatalografica

%DEDICATORIA
\frontmatter
% Texto primeiro autor 
\dedicatoriaautori{Aos meus irmãos, Jefferson e Jéssica}

\dedicatoria

%AGRADECIMENTOS
% Texto primeiro autor 
\agradecimentosautori{
	Agradeço imensamente ao professor Eduardo Stockler Tognetti, sua orientação sempre disposta guiou-me muito sabiamente desde o início deste trabalho, dois anos atrás. Sua enorme paciência tornou este projeto possível e a pesquisa em que me permitiu trabalhar proporcionou os conhecimentos mais valiosos que adquiri durante a graduação. A todos os professores do curso de Engenharia Mecatrônica da UnB, pelo esforço e dedicação nas aulas, obrigado!
	
	Agradeço a meu pai, Evandro Lourenço Rocha, minha mãe, Izabel Cristina Moraes Rosa Rocha, e aos meus amados irmãos, Jefferson Moraes Rocha e Jéssica Cristina Moraes Rocha, pelo amor incondicional e pelo apoio a todas as minhas decisões e escolhas desde sempre. A vocês dedico este trabalho e tudo que conquistei na vida.
	
	Deixo meus sinceros agradecimentos aos meus amigos Yuri Rocha, Henrique Balbino, Cristiana Miranda, De Hong Jung e a todos os outros que estiveram comigo todo este tempo. Obrigado pelo apoio de vocês sempre que demonstrei qualquer sinal de abatimento e fraqueza.
	
	Um abraço especial a Raphael Arthur, meu irmão, cuja amizade sincera e parceria incondicional é uma das recompensas mais valiosas que levarei da faculdade. Obrigado, irmão, por nunca me deixar desistir e pela absurda paciência que demandei de você! Espero continuar merecendo seu apoio sempre.
	
	Um agradecimento sincero a todos os membros do LARA, Laboratório de Automação e Robótica da UnB, aos companheiros da equipe de robótica (UnBeatables) e aos amigos da Moringa Digital, por serem minha grande motivação durante este período.
	
	Uma gratidão sincera à todos os amigos e professores do Colégio Protágoras, em especial a Marcos Araújo, Osmair e Permínio. Vocês me estenderam a mão e me proporcionaram a oportunidade de realizar este curso. É de vocês todo mérito, pelo esmero em transmitir conhecimento e por insistirem em me enviar para a UnB quando eu era contra a ideia de deixar minha família em Goiânia. Obrigado e, como sempre, vocês estavam certos! 
	
	Um agradecimento especial à Chiara Xavier. Sua amizade genuína é uma joia preciosa que me motiva a seguir em frente. Você me inspira a melhorar sempre. Obrigado, Chi, pela enorme paciência e pelo apoio sincero.
	
	Agradeço a todos que me apoiaram desde sempre e contribuíram  para que eu chegasse aqui hoje: os amigos do CPMG Hugo de Carvalho Ramos e o pessoal do Colégio Interação, especialmente o Professor Luís. A todos que tentaram me motivar de alguma forma, sinceramente, obrigado.
} 
% Conteudo da segunda pagina (caso agradecimentos sejam muito extensos)
\agradecimentosautorcont{}
% Texto segundo autor
\agradecimentosautorii{}
% Texto terceiro autor
\agradecimentosautoriii{}

\agradecimentos

\selectlanguage{english}%
%PORTUGUESE
\resumo{Resumo}
{
	\indent Plantas descritas matematicamente por sistemas não-lineares apresentam desafios para modelagem de sistemas e para aplicação de técnicas de controle convencionais. O procedimento mais simples nestas situações é aproximá-los a um estado local, linearizado, e assumir este comportamento pontual como o global do sistema. Sabe-se que esta abordagem fornece resultados que se afastam dos reais à medida que o estado do sistema distoa daquele ponto de referência. Abordagens fuzzy, como os modelos propostos por Takagi-Sugeno, são alternativas geralmente mais eficazes para solução deste problema, uma vez que fazem uma interpolação de várias modelagens em diversos pontos locais.
	
	Este trabalho faz uso da lógica fuzzy, utilizando modelos Takagi-Sugeno, para desenvolver controladores para uma planta de quatro-tanques, um sistema não-linear com acoplamento entre suas variáveis. O objetivo final é implementar este controlador em um CLP industrial e observar seu desempenho.
}
%ENGLISH
\vspace*{2cm}
\resumo{Abstract}
{
	\indent Plants mathematically described by non-linear systems present challenges for system modeling and for applying conventional control techniques. The simplest procedure in these situations is to approximate these systems to a linear local state and assume this punctual behavior as a global system. It is known that this approach yields results that deviate from the real ones as the state of the system moves away from that reference point. Fuzzy approaches, such as the models proposed by Takagi-Sugeno, are generally more effective alternatives to solve this problem, since they interpolate several models in several local points.
	
	This work makes use of fuzzy logic, using Takagi-Sugeno models, to develop controllers for a four-tank plant, a nonlinear system with degrees of coupling between its variables. The final goal is to implement this controller in an industrial PLC and observe its performance.
}\selectlanguage{brazil}%



\sumario

\listadefiguras

\listadetabelas

\selectlanguage{english}%

\chapter*{Lista de Símbolos}

\subsection*{Símbolos Latinos}

\begin{tabular}{>{\centering}p{0.1\textwidth}l}
CLP & Controlador Lógico Programável\tabularnewline
\foreignlanguage{brazil}{} & \selectlanguage{brazil}%
\selectlanguage{brazil}%
\tabularnewline
\foreignlanguage{brazil}{} & \selectlanguage{brazil}%
\selectlanguage{brazil}%
\tabularnewline
\foreignlanguage{brazil}{} & \selectlanguage{brazil}%
\selectlanguage{brazil}%
\tabularnewline
\foreignlanguage{brazil}{} & \selectlanguage{brazil}%
\selectlanguage{brazil}%
\tabularnewline
\end{tabular}

\subsection*{Símbolos Gregos}

\begin{tabular}{>{\centering}p{0.1\textwidth}l}
$\gamma_i$ & Abertura da válvula $i$\tabularnewline
\foreignlanguage{brazil}{} & \selectlanguage{brazil}%
\selectlanguage{brazil}%
\tabularnewline
\foreignlanguage{brazil}{} & \selectlanguage{brazil}%
\selectlanguage{brazil}%
\tabularnewline
\foreignlanguage{brazil}{} & \selectlanguage{brazil}%
\selectlanguage{brazil}%
\tabularnewline
\foreignlanguage{brazil}{} & \selectlanguage{brazil}%
\selectlanguage{brazil}%
\tabularnewline
\foreignlanguage{brazil}{} & \selectlanguage{brazil}%
\selectlanguage{brazil}%
\tabularnewline
\end{tabular}\selectlanguage{brazil}%



\selectlanguage{english}%

\chapter*{Notação}

Neste trabalho utiliza-se as denominações lógica fuzzy, lógica nebulosa e lógica difusa como sinônimos.

\begin{tabular}{>{\centering}p{0.1\textwidth}l}
	\foreignlanguage{brazil}{$M'$} & \selectlanguage{brazil} Transposta da matriz M $i$%
	\tabularnewline
	\foreignlanguage{brazil}{LMI} & \selectlanguage{brazil} Linear Matricial Inequalities (Desigualdade Linear Matricial)%
	\tabularnewline
	\foreignlanguage{brazil}{TS} & \selectlanguage{brazil} Sistema Takagi Sugeno%
	\tabularnewline
	\foreignlanguage{brazil}{} & \selectlanguage{brazil}%
	\selectlanguage{brazil}%
	\tabularnewline
	\foreignlanguage{brazil}{} & \selectlanguage{brazil}%
	\selectlanguage{brazil}%
	\tabularnewline
\end{tabular}\selectlanguage{brazil}%

 \selectlanguage{brazil}%



%CORPO PRINCIPAL
\mainmatter 
\setcounter{page}{1} \pagestyle{plain} 

\selectlanguage{english}%

\chapter{Introdução} \label{capIntrod}
Desenvolver controladores para sistemas não-lineares é quase sempre uma tarefa dispendiosa e complexa. Para plantas multivariáveis este desafio é ainda maior. É por este motivo que é prática comum recorrer-se à linearização das equações que as descrevem, obtendo uma aproximação do sistema inicial num formato que se encaixa às teorias de controle convencionais.

Esta linearização simples, realizada por meio da série de Taylor, resulta numa aproximação excelente localmente. No entanto, à medida que as variáveis controladas e manipuladas se afastam deste ponto de operação, condição na qual foi realizada a linearização, o modelo passa a se afastar da planta real.

Neste cenário, a abordagem fuzzy figura como excelente ferramenta para solução destes desvios. Aparecendo pela primeira vez nos trabalhos do professor Zadeh \cite{zadeh}, foi desenvolvida para aplicações em modelagem de sistemas nos trabalhas de Takagi e Sugeno\cite{takagiSugeno}. Seus métodos consistem na linearização convencional do sistema em múltiplos pontos escolhidos criteriosamente, baseados em um conjunto de métricas relevantes para o problema em questão. A partir daí desenvolve-se um conjunto de regras para determinar o grau de conformidade de cada estado do sistema à cada um dos pontos pré-modelados. Utiliza-se então como modelo a soma ponderada (uma interpolação) dos múltiplos modelos iniciais por estes cada coeficiente de pertinência. 

O objeto de estudo deste trabalho é o sistema de quatro tanques, desenvolvido por Karl Johansson \cite{johansson2} com o objetivo didático de demonstrar de forma ilustrativa conceitos e propriedades de sistemas com múltiplas entradas e saídas (MIMO, do inglês \textit{Multiple Input, Multiple Output}). Ele consiste em quatro tanques interconectados, um reservatório inferior, duas válvulas esferas e duas bombas de corrente contínua que bombeiam o fluido do reservatório inferior para os tanques de forma cruzada, de acordo com a razão entre os fluxos definida pela posição das válvulas. O sistema de quatro tanques é não linear. Seu modelo linearizado apresenta um zero multivariável que pode estar localizado tanto no semi-plano esquerdo quanto no  direito dependendo da configuração das válvulas. A abertura delas determina se o sistema é de fase mínima ou de fase não-mínima afetando a dinâmica geral entre entradas e saídas.

O objetivo é obter um controlador fuzzy, baseado no modelo Takagi-Sugeno, capaz de controlar os níveis do fluido nos tanques inferiores 1 e 2. As variáveis manipulados do processo são somente as tensões de entrada das bombas, que influenciam de maneira proporcionalmente direta no fluxo.

\section{Organização do Trabalho}
Os capítulos iniciais deste trabalham tratam da teoria fuzzy e sua aplicação em sistemas controlados. Já os capítulos finais aplicam essa teoria diretamente sobre a bancada de quatro-tanques e por meio de LMIs são desenvolvidos controladores para ela. No \jhhref{capDescSis}{capítulo} são apresentados a planta estudada e o CLP Rockwell onde os algoritmos são implementados. Em seguida, o \jhhref{capFundFuzzy}{capítulo} apresenta uma introdução aos conceitos da lógica e modelagem fuzzy e como aplicá-los. No \jhhref{capMod}{capítulo} são realizadas as três formas de modelagem do sistema abordadas neste trabalho: não-linear, linear e o modelo Takagi-Sugeno. No \jhhref{capControle}{capítulo} o projeto do controlador é desenvolvido, seguindo os conceitos de estabilidade baseados em desigualdade lineares matriciais. O \jhhref{capImp}{capítulo} apresenta a implementação dos algoritmos no CLP, utilizando as linguagens de programação aceitas por este. No \jhhref{capRes}{capítulo} são apresentados os resultados das simulações e do sistema real. Por fim as considerações finais são apresentadas no \jhhref{capConclusao}{capítulo}.


\selectlanguage{brazil}%



\selectlanguage{english}%

\chapter{Descrição do Sistema}


\selectlanguage{brazil}%



\selectlanguage{english}%

\chapter{Fundamentos Fuzzy}
A lógica fuzzy, ou difusa, foi introduzida originalmente por Zadeh, em seu artigo "Fuzzy Sets" \cite{zadeh}. Sua teoria de conjunto diverge da booleana 
no tratamento dos valores lógicos das variáveis, podendo assumir qualquer valor entre 0 e 1. 

\section{Conjuntos Fuzzy}
\indent De acordo com a teoria de conjuntos clássica, um elemento $x$ qualquer, pode pertencer ou não à um conjunto universo de discurso $U$, $x \in U$. Chamando de $f_u(x)$ a função de pertinência de $x$ ao conjunto $U$. Desta forma, tem-se:

\begin{align}
	f_u(x) : U \rightarrow {0,1}
	&& f_u(x) =
	\begin{cases*}
		1 & se e somente se $x \in U$ \\
		0 & caso contrário
	\end{cases*}
	\label{eqFPertinencia}
\end{align}

Essa definição binária se encaixa bem em problemas restritos, cujo caráter dos sistemas reflita essa separação de estados, por exemplo a paridade ou não de uma das somas dos bits de uma mensagem binária. No entanto, grande parte dos sistemas estudados nas teorias de controle trabalha com grandezas que possuem limites não tão claros assim, como exemplo a temperatura. Apesar de ser matematicamente bem definida, existem descrições como "frio" e "quente" que não podem ser representadas com este conjunto binário, uma vez que são conceitos vagos e imprecisos. A abordagem fuzzy é capaz de tratar a pertinência nestes casos, de onde vem a origem de seu nome "difusa".

\section{Funções de Pertinência}
	\subsection{Variáveis Linguísticas}
	
\section{Inferência}

\subsection{Fuzzyficacão}

\subsection{Regras}
	\subsection{Defuzzyficacão}


\section{Modelo Fuzzy Takagi-Sugeno}


\selectlanguage{brazil}%



\selectlanguage{english}%

\chapter{Controle Fuzzy} \label{capControle}
A sintonia de controladores para os modelos fuzzy deve garantir estabilidade no sistema final, formado pelo conjunto convexo dos pontos do modelo Takagi-Sugeno. Neste trabalho segue-se a mesma metodologia aplicada por Mozelli \cite{mozelli}, utilizando LMIS (\textit{Desigualdades Lineares Matriciais}), baseadas no método de Lyapunov \cite{lyapunov}, para a sintonia dos ganhos utilizados em malha fechada.

\section{Método de Lyapunov}
O método direto de Lyapunov é baseado na positividade de funções. Segue um embasamento para esta última:

\begin{mydef}
Uma função escalar contínua $w: \mathbb{R}^n \rightarrow \mathbb{R}$, $w(0) = 0$ é semidefinida positiva se, e somente se

	\begin{equation}
		w(x) \geq 0, \forall \ \ x \in \mathbb{R}^n - {0}
	\end{equation}
	Caso desigualdade seja estrita, então w será definida positiva. Uma função \textbf{g} é dita semidefinida (definida) negativa caso \textbf{-g} seja semidifenida (definida) positiva
\end{mydef}

O método de Lyapunov baseia-se no teorema a seguir:

\begin{myteo} \label{teoLyapunov}
	Um sistema dinâmico autônomo é globalmente estável se existe uma função escalar $V : \mathbb{R}^n \rightarrow \mathbb{R} $ tal que:
	
	\begin{itemize}
		\item V é definida positiva
		\item V possui derivada de primeira ordem
		\item $\dot{V}$ é definida negativa
		\item $V \rightarrow \infty$ a medida em que $\|x\| \rightarrow \infty$
	\end{itemize}
\end{myteo}

Assim, uma função V que satisfaça todo os requisitos do Teorema 1 para um dado sistema é chamada de função de Lyapunov. 


\section{Estabilidade Fuzzy}
Para um modelo TS, simular ao na \jhhref{eqModTakSug}{Equação}, tem-se:
\begin{align*}
	\dot{x}(t) = \frac{\sum_{i=1}^{4}  w_i(c(t))(A_i  x(t) +  B_i  u(t))}{\sum_{j=1}^{4} w_j(c(t))} \vspace{0.5cm}
\end{align*}

Simplificando os termos de ativação de forma a obter um parâmetro A(c), definido como:
\begin{align*}
	\alpha_i (c(t)) &:= \frac{w_i(c(t))}{\sum_{j=1}^{r}w_j(c(t))} \\
	A(\alpha) &:= \sum_{i=1}^{r} \alpha_i (c(t))A_i
\end{align*}

Aplicando-se a Teoria de Lyapunov, como proposto por Tanaka e Wang \cite{wang}, uma vez que o modelo final é convexo, basta tratar a \jhhref{eqLyapXk}{Equação} nos vértices do sistema, ou seja, em cada um dos sistemas que compõem as \jhhref{eqRegraIGeral}{Regras}. 
A análise computacional via LMIs possibilita a busca por uma matriz P que satisfaça essas condições, caso ela exista, prova-se a estabilidade do sistema. Assim, o sistema TS é globalmente assintoticamente estável \cite{tanakaWang} existe solução para:

\begin{align} \label{eqLyapXk}
	encontre \ \ &P \nonumber \\
	s.a \ \ &P \succ 0 \nonumber \\
	&A_i'P + P A_i \prec 0, \ \ i=1,2,3, ... , r
\end{align}

Seguindo estas premissas \cite{wang}, obtém-se o seguinte teorema:
\begin{myteo} \label{teoControlador}
Dado um sistema fuzzy Takagi-Sugeno, convexo, composto por r regras e modelos correspondentes, em malha fechada com ganhos $K_i = M_i X^{-1}$, sua estabilidade é verificada se existe solução para o problema
	\begin{align} \label{eqContFuzzy}
		encontre \ \ &X = X', \ \ i = 1,2,...,r \nonumber \\
		&X \succ 0 \nonumber \\
		s.a \ \ & 
		\begin{bmatrix}
			X	&	XA_i' - M_i'B_i'  \\
			A_iX - B_iM_i &	X
		\end{bmatrix} \succ 0 \nonumber \\
	\end{align}
\end{myteo}

\selectlanguage{brazil}%



\selectlanguage{english}%

\chapter{Modelagem}
\label{capMod}
Os modelos físicos do sistema de quatro-tanques utilizados neste trabalho são apresentados nas seções a seguir.

\section{Modelo Não Linear}
Baseado nos princípios de conservação de massa e na lei de Bernoulli para líquidos incompressíveis tem-se o seguinte sistema de equações não lineares que descrevem o processo.

\begin{equation} \label{eqModNL}
	\begin{cases}
		\dot{h_{1}} = \frac{1}{A_{1}}(a_{3}\sqrt{2gh_{3}} + \gamma_{1}k_{1}v_{1} - a_{1}\sqrt{2gh_{1}})\\
		
		\dot{h_{2}} = \frac{1}{A_{2}}(a_{4}\sqrt{2gh_{4}} + \gamma_{2}k_{2}v_{2} - a_{2}\sqrt{2gh_{2}})\\
		
		\dot{h_{3}} = \frac{1}{A_{3}}((1 - \gamma_{2})k_{2}v_{2} - a_{3}\sqrt{2gh_{3}})\\
		
		\dot{h_{4}} = \frac{1}{A_{4}}((1 - \gamma_{1})k_{1}v_{1} - a_{4}\sqrt{2gh_{4}})
	\end{cases}
\end{equation}

Na \jhhref{eqModNL}{Equação}, os termos $h_{i}$, $A_{i}$ e $a_{i}$ representam o nível de água, a área da secção transversal da base do tanque $i$ e a área de secção transversal do orifício de saída do tanque $i$, $i=1,2,3,4$, respectivamente. A constante de fluxo da bomba $j$ e a tensão aplicada sobre ela são dadas por $k_{j}$ e $v_{j}$, $j=1,2$. O parâmetro $\gamma_{1}$ é a razão entre os fluxos para os tanques 1 e 4 enquanto $\gamma_{2}$ é a razão entre os fluxos para os tanques 2 e 3 e g é a aceleração da gravidade. 

É fácil notar nas equações deste sistema os termos não lineares (as raízes). O aspecto do acoplamento entre as variáveis também pode ser observado ao analisar as equações: o nível $h_1$ varia conforme o fluxo da bomba 1, dependente de $v_1$, e conforme o nível $h_3$, que por sua vez depende do fluxo da bomba 2, $v_2$. Assim, as variáveis manipuladas, $v_1$ e $v_2$, influem em ambos os níveis simultaneamente, apresentando um desafio considerável para a estabilização dos níveis desejados, $h_1$ e $h_2$, que são as variáveis controladas deste trabalho.

\section{Linearização}
Linearizando o sistema em torno dos ponto de operação $\overline{h}=(\overline{h_{1}},\overline{h_{2}},\overline{h_{3}},\overline{h_{4}})$ e $\overline{v}=(\overline{v_{1}},\overline{v_{2}})$, por expansão em série de Taylor, obtém-se a seguinte representação no espaço de estados:

\begin{multline} \label{eqModLinear}
	\begin{bmatrix}
		\dot{\Delta h_{1}} \\
		\dot{\Delta h_{2}} \\
		\dot{\Delta h_{3}} \\
		\dot{\Delta h_{4}} 
	\end{bmatrix}
	= 
	\begin{bmatrix}
		\frac{-a_{1}\sqrt{2g}}{2A_{1}\sqrt{h_{1}}} & 0 & \frac{a_{3}\sqrt{2g}}{2A_{1}\sqrt{h_{3}}} & 0 \\
		0 & \frac{-a_{2}\sqrt{2g}}{2A_{2}\sqrt{h_{2}}} & 0 & \frac{a_{4}\sqrt{2g}}{2A_{2}\sqrt{h_{4}}} \\
		0 & 0 & \frac{-a_{3}\sqrt{2g}}{2A_{3}\sqrt{h_{3}}} & 0 \\
		0 & 0 & 0 & \frac{-a_{4}\sqrt{2g}}{2A_{4}\sqrt{h_{4}}}
	\end{bmatrix}
	\begin{bmatrix}
		\Delta h_{1} \\
		\Delta h_{2} \\
		\Delta h_{3} \\
		\Delta h_{4} 
	\end{bmatrix}
	+
	\begin{bmatrix}
		\frac{\gamma_{1}k_{1}}{A_{1}} & 0 \\
		0 & \frac{\gamma_{2}k_{2}}{A_{2}} \\
		0 & \frac{(1-\gamma_{2}) k_{2}}{A_{3}} \\
		\frac{(1-\gamma_{1})k_{1}}{A_{4}} & 0
	\end{bmatrix}
	\begin{bmatrix}
		\Delta v_{1} \\
		\Delta v_{2}
	\end{bmatrix}
\end{multline}
\begin{equation}
	\begin{bmatrix}
		y_{1} \\
		y_{2} \\
		y_{3} \\
		y_{4} 
	\end{bmatrix}
	= 
	I
	\begin{bmatrix}
		\Delta h_{1} \\
		\Delta h_{2} \\
		\Delta h_{3} \\
		\Delta h_{4} 
	\end{bmatrix}
	\label{eq3}
\end{equation}

em que $y_{i}$ são as saídas medidas do sistema, $\Delta h_{i}=h_{i} - \overline{h_{i}}$, $\Delta v_{i}=v_{i} - \overline{v_{i}}$, e $i=1,2,3,4$.

E por fim, a matriz função de transferência do sistema obtida é:
\begin{equation}
		G(s) = 
	\begin{bmatrix}
		\frac{T_{1}\gamma_{1}k_{1}}{A_{1}(1+sT_{1})} &  \frac{T_{1}(1-\gamma_{2})k_{2}}{A_{1}(1+sT_{3})(1+sT_{1})} \\
		\frac{T_{2}(1-\gamma_{1})k_{1}}{A_{2}(1+sT_{4})(1+sT_{2})} &  \frac{T_{2}\gamma_{2}k_{2}}{A_{2}(1+sT_{2})} \\
		0 &  \frac{T_{3}(1-\gamma_{2})k_{2}}{A_{3}(1+sT_{3})} \\
		\frac{T_{4}(1-\gamma_{1})k_{1}}{A_{4}(1+sT_{4})} &  0 
	\end{bmatrix} 
	\label{eq4}
\end{equation}

em que $G(s)=\frac{\Delta h(s)}{\Delta v(s)}$ e $T_{i}=\frac{2A_{i}\sqrt{h_{i}}}{a_{i}\sqrt{2g}}$, $i=1,2,3,4$.

\section{Modelagem Fuzzy Takagi-Sugeno} \label{secModFuzzy}
A modelagem via Takagi-Sugeno segue os mesmos passos propostos na \jhhref{secTakSug}{Seção}: escolhem-se as variáveis linguística do sistema e seus conjuntos fuzzy, definem-se as funções de pertinência de cada um deles, as regras Se-Então, a fórmula de ativação e o resultado final do modelo.

\subsection{Variáveis Linguísticas}
Como visto, as regras do modelo são ativadas de acordo com o estado atual do sistema. As variáveis aferidas da planta são os níveis, por este motivo serão as variáveis linguísticas definidas para o modelo fuzzy. Escolhem-se então "Nível do Tanque 1" e "Nível do Tanque 2" e os conjuntos {nível baixo, nível alto} para cada uma.

\subsection{Pertinência}
Como visto, a linearização é baseada nos estados estacionários do sistema, ou seja, são escolhidos pontos em que $\dot{h}(t) = 0$. Assim, o modelo linear trata do valor de desvio das variáveis e é dado por: 

	\begin{equation}
		\dot{\Delta h}(t) =  A \Delta h(t) +  B \Delta u(t)
	\end{equation}

Seguindo os passos descritos na \jhhref{secFncPert}{Seção} definem-se os conjuntos escolhidos. Há quatro vértices para os estados do sistema:
\begin{align}
	\begin{cases}
		\text{Nível 1 Baixo}
		&\begin{cases}
			\text{Nível 2 Baixo}\\
			\text{Nível 2 Alto}
		\end{cases}	\\
		\text{Nível 1 Alto}
		&\begin{cases}
			\text{Nível 2 Baixo}\\
			\text{Nível 2 Alto}
		\end{cases}
	\end{cases}
\end{align}
Obtemos um modelo linear em cada um destes vértices:
	\begin{align}
	\dot{\Delta h}(t) =  A_i \Delta h(t) +  B_i \Delta u(t) \\
	i = 1,2,3,4
	\end{align}


Como descrito na \jhhref{tabDescPlanta}{Tabela}, a altura dos tanques é de 25cm. Assim, escolhe-se os limitantes 5cm como "completamente" verdade para o nível baixo e  25cm como "completamente" verdade para o nível alto. As funções de pertinência obtidas são:


\begin{figure}[H]
	\centering
	\includegraphics[width=0.7\textwidth]{img/pert_niveis.png}
	\caption{\label{figPertMod} Funções de Pertinência.}
\end{figure}

Onde $M_1(h_1(t))$ e $M_2(h_1(t))$ é o grau de pertinência do nível $h_1(t)$ aos conjuntos "baixo" e "alto", respectivamente. De maneira análoga, tem-se $N_1(h_2(t))$ e $N_2(h_2(t))$ para o nível $h_2(t)$.

\subsection{Regras Se-Então}
Haverá uma regra para cada um dos sistemas dados:

	\begin{itemize}
		\item Regra 1:\\ 
		\begin{align*}
			\begin{cases}
				\text{SE } &h_1 \text{ é baixo e } h_2 \text{ é baixo} \\
				\text{ENTÃO } &\dot{\Delta h}(t) =  A_1 \Delta h(t) +  B_1 \Delta u(t)
			\end{cases}		
		\end{align*}

		\item Regra 2:\\ 
		\begin{align*}
		\begin{cases}
			\text{SE } &h_1 \text{ é baixo e } h_2 \text{ é alto} \\
			\text{ENTÃO } &\dot{\Delta h}(t) =  A_2 \Delta h(t) +  B_2 \Delta u(t)
		\end{cases}		
		\end{align*}

		\item Regra 3:\\ 
		\begin{align*}
		\begin{cases}
			\text{SE } &h_1 \text{ é alto e } h_2 \text{ é baixo} \\
			\text{ENTÃO } &\dot{\Delta h}(t) =  A_3 \Delta h(t) +  B_3 \Delta u(t)
		\end{cases}		
		\end{align*}

		\item Regra 4:\\ 
		\begin{align*}
		\begin{cases}
			\text{SE } &h_1 \text{ é alto e } h_2 \text{ é alto} \\
			\text{ENTÃO } &\dot{\Delta h}(t) =  A_4 \Delta h(t) +  B_4 \Delta u(t)
		\end{cases}		
		\end{align*}
	\end{itemize}

\subsection{Ativação}
O nível de ativação de cada uma das \textbf{Regras} $i$ é dado respectivamente por $w_i$:
	\begin{equation}
	\begin{aligned}
		w_{1}(t) = M_1(h_1(t)) * N_1(h_2(t)) \\
		w_{2}(t) = M_1(h_1(t)) * N_2(h_2(t)) \\
		w_{3}(t) = M_2(h_1(t)) * N_1(h_2(t)) \\
		w_{4}(t) = M_2(h_1(t)) * N_2(h_2(t))
	\end{aligned}
	\label{eqGrauAtiv4T}
	\end{equation}
	
\subsection{Modelo Final}
O modelo Takagi-Sugeno fornece, finalmente:
	\begin{align} \label{eqTakSugPlanta}
		\dot{\Delta h}(t) = \frac{\sum_{i=1}^{4}  w_i(h(t))(A_i \Delta h(t) +  B_i \Delta u(t))}{\sum_{i=1}^{4} w_i(h(t))}
	\end{align}

\subsection{Controlador Fuzzy}
No \jhhref{capControle}{capítulo} demonstra-se o projeto de controladores capazes de estabilizar o sistema em todos os pontos de \ref{eqTakSugPlanta}. O desenvolvimento do controlador final para modelo segue os mesmos passos já apresentados: realiza-se a sintonização do ganho $K$ a ser utilizado para cada regra e o grau de ativação de cada ganho é obtido pelas mesmas \jhhref{eqGrauAtiv4T}{equações}.
 O ganho final a ser utilizado é dado por:
	\begin{align} \label{eqModContFuzzy}
		K = \frac{\sum_{i=1}^{4}  w_i(h(t))K_i}{\sum_{i=1}^{4} w_i(h(t))}
	\end{align}

Deseja-se neste trabalho obter um controlador capaz de prover erro nulo para ambas os níveis inferiores em estado estacionário. Desta forma, foi desenvolvido um projeto aumentado dos erros integrais das variáveis controlados. O modelo final dos sistema é apresentado na \jhhref{figSimPlantCtrl}{figura} a seguir:

\begin{figure}[H]
	\begin{centering}
		\includegraphics[width=\textwidth]{img/modelo_controlado.png}
		\par\end{centering}
	\caption{\label{figSimPlantCtrl}Espaço de estados da planta controlada}
\end{figure}

\selectlanguage{brazil}%



\selectlanguage{english}%

\chapter{Implementação}
\label{capImp}
\section{Identificação}

\section{CLP}

\selectlanguage{brazil}%



\selectlanguage{english}%

\chapter{Resultados} \label{capRes}

O modelo fuzzy desenvolvido a partir das seções anteriores foi simulado utilizando o software MATLAB \cite{matlab} e implementado na bancada real via CLP Rockwell. As seções a seguir apresentam os resultados obtidos em cada caso.

\section{Simulações} \label{secAnalise}
A planta de quatro-tanques, como apresentada no \jhhref{capDescSis}{capítulo}, compõe um sistema capaz de ilustrar diversas dinâmicas para suas variáveis de processo. Assim, são escolhidas uma configuração em \textbf{fase mínima }e outra em \textbf{fase não-mínima} e a partir delas a modelagem e o controlador fuzzy são desenvolvidos. 

\subsection{Fase Mínima}
Nesta configuração a maior parte do fluído que saí das bombas é direcionado diretamente para os tanques controlados, ou seja $\gamma_i > 0.5$. A \jhhref{tabFaseMinima}{tablea} a seguir apresenta suas especificações.

\begin{center} \label{tabFaseMinima}
	\begin{tabular}{|c|c|}
		\hline
		\multicolumn{2}{|c|}{Especificações Iniciais da Planta} \\
		\hline
		A1, A3 $(cm^2)$ & 28 \\ \hline
		A2, A4 $(cm^2)$ & 32 \\ \hline
		a1, a3 $(cm^2)$ & 0.071 \\ \hline
		a2, a4 $(cm^2)$ & 0.057 \\ \hline
		g $cm/s$ & 981 \\ \hline
		k1 & 3,33 \\ \hline
		k2 & 3.35 \\ \hline
		$\gamma_1$ & 0.70 \\ \hline
		$\gamma_2$ & 0.60 \\ \hline
		\hline
	\end{tabular}
\end{center}

O \jhhref{eqModNL}{modelo não linear} para esta configuração é:
\begin{equation}
\begin{cases}
\dot{h_{1}} = \frac{1}{A_{1}}(a_{3}\sqrt{2gh_{3}} + \gamma_{1}k_{1}v_{1} - a_{1}\sqrt{2gh_{1}})\\

\dot{h_{2}} = \frac{1}{A_{2}}(a_{4}\sqrt{2gh_{4}} + \gamma_{2}k_{2}v_{2} - a_{2}\sqrt{2gh_{2}})\\

\dot{h_{3}} = \frac{1}{A_{3}}((1 - \gamma_{2})k_{2}v_{2} - a_{3}\sqrt{2gh_{3}})\\

\dot{h_{4}} = \frac{1}{A_{4}}((1 - \gamma_{1})k_{1}v_{1} - a_{4}\sqrt{2gh_{4}})
\end{cases}
\label{eqFMNL}
\end{equation}

Escolhendo os cojuntos fuzzy \{"baixo","alto"\} e definindo \{5 , 15\} como seus representantes os níveis 1 e 2, por combinação simples obtém-se os seguintes pontos de linearização:
\begin{center}
	\begin{tabular}{|c|c|c|c|c|c|c|}
		\hline
		Sistema & Nível 1 ($\bar{h_1}$) & Nível 2 ($\bar{h_2}$) & Nível 3 ($\bar{h_3}$) & Nível 1 ($\bar{h_4}$) & Tensão 1 ($\bar{v_1}$) & Tensão 2 ($\bar{v_2}$) \\ \hline
		1 & 5 & 5 & 10 & 10 & 10 & 10 \\ \hline
		2 & 5 & 15 & 10 & 10 & 10 & 10 \\ \hline
		3 & 15 & 5 & 10 & 10 & 10 & 10 \\ \hline
		4 & 15 & 15 & 10 & 10 & 10 & 10 \\	\hline
	\end{tabular}
\end{center}

Haverá então quatro regras Se-Então para composição do modelo TS final. As imagens a seguir apresenta a comparação entre os modelos \jhhref{eqModNL}{não-linear}, \jhhref{eqModLinear}{linearizado} em um ponto único e \jhhref{eqTakSugPlanta}{Takagi-Sugeno}:

\begin{figure}[H]
	\centering
	\includegraphics[width=0.7\textwidth]{img/FM_h1_5_10_15.png}
	\caption{\small Linearização Convencional: $ \bar{h1}=5, \bar{h2}=5$. Linearizações Fuzzy: $\bar{h1}=[10 \ \ 15] \ \ \bar{h2}=[10 \ \ 15]$ }
	\label{figH1TS2}
\end{figure}

\begin{figure}[H]
	\centering
	\includegraphics[width=0.7\textwidth]{img/FM_h1_5_10_15.png}
	\caption{\small Linearização Convencional: $ \bar{h1}=5, \bar{h2}=5$. Linearizações Fuzzy: $\bar{h1}=[10 \ \ 15] \ \ \bar{h2}=[10 \ \ 15]$ }
	\label{figH2TS2}
\end{figure}

É notável que o modelo fuzzy representa de modo mais eficiente o sistema. Como dito, o modelo TS pode se aproximar o quanto se desejar do não-linear no qual se baseia. A \jhhref{imgTS5}{imagem} apresenta um modelo com 5 conjuntos para os dois níveis e a \jhhref{imgTS15}{imagem} utilizando 15. É importante notar, no entanto, que a complexidade do modelo é exponencial, devido a combinação dos conjuntos das variáveis linguísticas presentes, assim, o primeiro é composto por 25 regras Se-Então e o segundo por 225!

\begin{figure}[H]
	\centering
	\begin{tabular}{cc}
		\includegraphics[width=0.5\textwidth,keepaspectratio]{img/FM_h1_5_10_15.png} &
		\includegraphics[width=0.5\textwidth,keepaspectratio]{img/FM_h1_5_10_15.png} \\
		(a) Pertinência do conjunto "muito frio" &
		(b) Pertinência do conjunto "frio"
	\end{tabular}
	\caption{\label{imgTS5} Funções de Pertinência.}
\end{figure}

\begin{figure}[H]
	\centering
	\begin{tabular}{cc}
		\includegraphics[width=0.5\textwidth,keepaspectratio]{img/FM_h1_5_10_15.png} &
		\includegraphics[width=0.5\textwidth,keepaspectratio]{img/FM_h1_5_10_15.png} \\
		(a) Pertinência do conjunto "muito frio" &
		(b) Pertinência do conjunto "frio"
	\end{tabular}
	\caption{\label{imgTS15} Funções de Pertinência.}
\end{figure}

A partir das \jhhref{eqContFuzzy}{equações} são desenvolvidos os controladores para cada uma das regras. A tabela a seguir apresenta os ganhos obtidos:
\begin{center}
	\begin{tabular}{|c|c|}
		\hline
		Regra & Ganho \\ \hline
		 1 & $ K = 
			\begin{bmatrix}
				1 & 2 & 3 & 4 & 5 & 6 \\
				1 & 2 & 3 & 4 & 5 & 6
			\end{bmatrix}$ \\[20pt] \hline
		2 & $ K = 
			\begin{bmatrix}
				1 & 2 & 3 & 4 & 5 & 6 \\
				1 & 2 & 3 & 4 & 5 & 6
			\end{bmatrix}$ \\[20pt] \hline
		3 & $ K = 
			\begin{bmatrix}
				1 & 2 & 3 & 4 & 5 & 6 \\
				1 & 2 & 3 & 4 & 5 & 6
			\end{bmatrix}$ \\[20pt] \hline
		4 & $ K = 
			\begin{bmatrix}
				1 & 2 & 3 & 4 & 5 & 6 \\
				1 & 2 & 3 & 4 & 5 & 6
			\end{bmatrix}$ \\[20pt] \hline
	\end{tabular}
\end{center}

\subsection{Fase Não-Mínima}
A tabela a seguir apresenta as especificações do sistema, nota-se por $\gamma_1$ e $\gamma_2$ que o sistema está em fase não mínima.
\begin{center}
	\begin{tabular}{|c|c|}
		\hline
		\multicolumn{2}{|c|}{Especificações do sistema} \\
		\hline'
		A1, A3 $(cm^2)$ & 28 \\ \hline
		A2, A4 $(cm^2)$ & 32 \\ \hline
		a1, a3 $(cm^2)$ & 0.071 \\ \hline
		a2, a4 $(cm^2)$ & 0.057 \\ \hline
		g $(cm/s)$ & 981 \\ \hline
		k1 & 3,14 \\ \hline
		k2 & 3.29 \\ \hline
		$\gamma_1$ & 0.43 \\ \hline
		$\gamma_2$ & 0.34 \\ \hline
		\hline
	\end{tabular}
\end{center}

O \jhhref{eqModNL}{modelo não linear} para esta configuração é:
\begin{equation}
\begin{cases}
	\dot{h_{1}} = \frac{1}{A_{1}}(a_{3}\sqrt{2gh_{3}} + \gamma_{1}k_{1}v_{1} - a_{1}\sqrt{2gh_{1}})\\
	
	\dot{h_{2}} = \frac{1}{A_{2}}(a_{4}\sqrt{2gh_{4}} + \gamma_{2}k_{2}v_{2} - a_{2}\sqrt{2gh_{2}})\\
	
	\dot{h_{3}} = \frac{1}{A_{3}}((1 - \gamma_{2})k_{2}v_{2} - a_{3}\sqrt{2gh_{3}})\\
	
	\dot{h_{4}} = \frac{1}{A_{4}}((1 - \gamma_{1})k_{1}v_{1} - a_{4}\sqrt{2gh_{4}})
\end{cases}
\label{eqFNMNL}
\end{equation}

Escolhendo os cojuntos fuzzy \{"baixo","alto"\} e definindo {5,15} para os níveis 1 e 2, obtém-se, a partir da\jhhref{eqFMNL}{equação} os seguintes pontos de linearização:

\begin{center}
	\begin{tabular}{|c|c|c|c|c|c|c|}
		\hline
		Sistema & Nível 1 ($h_1$) & Nível 2 ($h_2$) & Nível 3 ($h_3$) & Nível 1 ($h_4$) & Tensão 1 ($v_1$) & Tensão 2 ($v_2$) \\ \hline
		1 & 5 & 5 & 10 & 10 & 10 & 10 \\ \hline
		2 & 5 & 15 & 10 & 10 & 10 & 10 \\ \hline
		3 & 15 & 5 & 10 & 10 & 10 & 10 \\ \hline
		4 & 15 & 15 & 10 & 10 & 10 & 10 \\	\hline
	\end{tabular}
\end{center}

Nas figuras que se seguem apresentam-se as respostas dos modelos à degraus aplicados ao sistema.  Observa-se que o modelo linear apresenta bons resultados quando o estado do sistema é próximo ao ponto de operação. Já para os modelos fuzzy, quanto mais pontos de linearização utilizados, melhor o resultado, embora mais complexo o custo computacional.

\begin{figure}[H]
	\centering
	\includegraphics[width=0.7\textwidth]{img/h1Fuz5_10.png}
	\caption{\small Linearização Convencional: $ \bar{h1}=5, \bar{h2}=5$. Linearizações Fuzzy: $\bar{h1}=[5 \ \ 10] \ \ \bar{h2}=[5 \ \ 10]$ }
	\label{figH1FNM_1}
\end{figure}


\section{Implementação}

\selectlanguage{brazil}%


%\selectlanguage{english}%

\chapter{Resultados} \label{capRes}
\section{Simulações} \label{secAnalise}
\subsection{Fase Mínima}

\subsection{Fase Não-Mínima}
\section{Implementação}

\selectlanguage{brazil}%



\selectlanguage{english}%

\chapter{Conclusão} \label{capConclusao}

O sistema fuzzy representa melhor a dinâmica geral da planta através da interpolação de vários modelos lineares. Os resultados obtidos demonstram isso e os controladores sintonizados apresentam comportamentos eficientes.

\selectlanguage{brazil}%



%BIBLIOGRAPHY
\addcontentsline{toc}{chapter}{Referências Bibliográficas}
\bibliographystyle{IEEEtran}
\bibliography{bibliography}

%ANEXOS
\appendix
\selectlanguage{english}%

\chapter{Códigos Utilizados} \label{anexoCodigos}

\section{Matlab}
	Planta Simulada:
	\lstinputlisting{./codigo/planta.m}
	
\section{Texto Estruturado}
	\lstinputlisting{./codigo/tg_strutcText.txt}

\selectlanguage{brazil}%



\end{document}
