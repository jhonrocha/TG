\selectlanguage{english}%

\chapter{Conclusão} \label{capConclusao}


Neste trabalho foi realizado o desenvolvimento de controladores fuzzy para o sistema de quatro-tanques. Inicialmente foram realizadas as modelagens não-linear e linear do sistema. Em seguida, partindo da teoria fuzzy e dos modelos Takagi-Sugeno, foram estudadas suas aplicações e o desenvolvimento sistemático de controladores via LMIs.

O CLP Rockwell foi instalado e integrado à bancada, juntamente com os módulos de entrada e saída de dados utilizados. Seguiu-se a integração aos softwares de desenvolvimento e as configurações de comunicação entre eles. Foram implementados os algoritmos para controle da banca.

Por fim, foram simulados os modelos TS, que demonstraram os resultados esperados. Em seguida, os controladores desenvolvidos a partir deles foram simulados sobre o modelo não linear e os resultados obtidos foram satisfatórios. A implementação em bancada e seus resultados foram observados.

\section{Objetivos}
O objetivo geral de desenvolver o controlador fuzzy foi obtido e testado, os resultados simulados dos modelos demonstraram comportamento eficiente, embora o sistema real tenha apresentado uma resposta não tão eficaz. A aplicação dos métodos propostos por Takagi-Sugeno à planta de quatro tanques foi realizada e a interpolação dos múltiplos ganhos utilizando as linguagens disponíveis no CLP Rockwell instalado foi desenvolvida.

\section{Trabalhos Futuros}
% TODO: Critérios de Desempenho
% TODO: Esse processo precisa ser  uma vez que o comportamento observado no sistema real parece não manter suas configurações dia a dia
Para trabalhos futuros indica-se a inclusão das restrições de entrada na sintonia baseada em LMI dos controladores. Como visto, o modelo TS pode aproximar-se tanto quanto se queira do real, assim, realizar-se a identificação do sistemas em vários pontos também tornaria o projeto do controlador mais preciso. A inclusão da zona morta da bomba no modelo poderia eliminar as oscilações observadas nos resultados.

\epigraph{Há momentos, e você chega a esses momentos, em que de repente o tempo para e acontece a eternidade.}{Fiodor Dostoievski}

\epigraph{texto}{referencia}
\epigraph{texto}{referencia}
\epigraph{texto}{referencia}

\selectlanguage{brazil}%

