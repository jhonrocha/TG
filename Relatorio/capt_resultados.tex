\selectlanguage{english}%

\chapter{Resultados} \label{capRes}

O modelo fuzzy desenvolvido a partir das seções anteriores foi simulado utilizando o software MATLAB \cite{matlab} e implementado na bancada real via CLP Rockwell. As seções a seguir apresentam os resultados obtidos em cada caso.

\section{Simulações} \label{secAnalise}
A planta de quatro-tanques, como apresentada no \jhhref{capDescSis}{capítulo}, compõe um sistema capaz de ilustrar diversas dinâmicas para suas variáveis de processo. Assim, são escolhidas uma configuração em \textbf{fase mínima }e outra em \textbf{fase não-mínima} e a partir delas a modelagem e o controlador fuzzy são desenvolvidos. 

\subsection{Fase Mínima}
Nesta configuração a maior parte do fluído que saí das bombas é direcionado diretamente para os tanques controlados, ou seja $\gamma_i > 0.5$. A \jhhref{tabFaseMinima}{tablea} a seguir apresenta suas especificações.

\begin{center} \label{tabFaseMinima}
	\begin{tabular}{|c|c|}
		\hline
		\multicolumn{2}{|c|}{Especificações Iniciais da Planta} \\
		\hline
		A1, A3 $(cm^2)$ & 28 \\ \hline
		A2, A4 $(cm^2)$ & 32 \\ \hline
		a1, a3 $(cm^2)$ & 0.071 \\ \hline
		a2, a4 $(cm^2)$ & 0.057 \\ \hline
		g $cm/s$ & 981 \\ \hline
		k1 & 3,33 \\ \hline
		k2 & 3.35 \\ \hline
		$\gamma_1$ & 0.60 \\ \hline
		$\gamma_2$ & 0.70 \\ \hline
		\hline
	\end{tabular}
\end{center}

O \jhhref{eqModNL}{modelo não linear} para esta configuração é:
\begin{equation}
\begin{cases}
\dot{h_{1}} = \frac{1}{A_{1}}(a_{3}\sqrt{2gh_{3}} + \gamma_{1}k_{1}v_{1} - a_{1}\sqrt{2gh_{1}})\\

\dot{h_{2}} = \frac{1}{A_{2}}(a_{4}\sqrt{2gh_{4}} + \gamma_{2}k_{2}v_{2} - a_{2}\sqrt{2gh_{2}})\\

\dot{h_{3}} = \frac{1}{A_{3}}((1 - \gamma_{2})k_{2}v_{2} - a_{3}\sqrt{2gh_{3}})\\

\dot{h_{4}} = \frac{1}{A_{4}}((1 - \gamma_{1})k_{1}v_{1} - a_{4}\sqrt{2gh_{4}})
\end{cases}
\label{eqFMNL}
\end{equation}

Escolhendo os cojuntos fuzzy \{"baixo","alto"\} e definindo \{5 , 15\} como seus representantes os níveis 1 e 2, por combinação simples obtém-se os seguintes pontos de linearização:
\begin{center}
	\begin{tabular}{|c|c|c|c|c|c|c|}
		\hline
		Sistema & Nível 1 ($\bar{h_1}$) & Nível 2 ($\bar{h_2}$) & Nível 3 ($\bar{h_3}$) & Nível 1 ($\bar{h_4}$) & Tensão 1 ($\bar{v_1}$) & Tensão 2 ($\bar{v_2}$) \\ \hline
		1 & 5 & 5 & 0.0334 & 2.9076 & 3.2321 & 0.5716 \\ \hline
		2 & 5 & 15 & 0.9431 & 1.1033 & 1.9910 & 3.0390 \\ \hline
		3 & 15 & 5 & 0.2229 & 13.0191 & 6.8393 & -1.4773 \\ \hline
		4 & 15 & 15 & 0.1001 & 8.7228 & 5.5982 & 0.9900 \\	\hline
	\end{tabular}
\end{center}

Haverá então quatro regras Se-Então para composição do modelo TS final. As imagens a seguir apresenta a comparação entre os modelos \jhhref{eqModNL}{não-linear}, \jhhref{eqModLinear}{linearizado} em um ponto único e \jhhref{eqTakSugPlanta}{Takagi-Sugeno}:

\begin{figure}[H]
	\centering
	\includegraphics[width=\textwidth]{img/h1_ts2.png}
	\caption{\small Nível do Tanque 1}
	\label{figH1TS2}
\end{figure}

\begin{figure}[H]
	\centering
	\includegraphics[width=\textwidth ]{img/h2_ts2.png}
	\caption{\small Nível do Tanque 2}
	\label{figH2TS2}
\end{figure}

É notável que o modelo fuzzy representa de modo mais eficiente o sistema. Como dito, o modelo TS pode se aproximar o quanto se desejar do não-linear no qual se baseia. A \jhhref{imgTS5}{imagem} apresenta um modelo com 5 conjuntos para os dois níveis e a \jhhref{imgTS15}{imagem} utilizando 15. É importante notar, no entanto, que a complexidade do modelo é exponencial, devido a combinação dos conjuntos das variáveis linguísticas presentes, assim, o primeiro é composto por 25 ($5^2$) regras Se-Então e o segundo por 225 ($15^2$).

\begin{figure}[H]
	\centering
	\begin{tabular}{cc}
		\includegraphics[width=0.5\textwidth,keepaspectratio]{img/h1_ts5.png} &
		\includegraphics[width=0.5\textwidth,keepaspectratio]{img/h2_ts5.png} \\
		(a) Pertinência do conjunto "muito frio" &
		(b) Pertinência do conjunto "frio"
	\end{tabular}
	\caption{\label{imgTS5} Funções de Pertinência.}
\end{figure}

\begin{figure}[H]
	\centering
	\begin{tabular}{cc}
		\includegraphics[width=0.5\textwidth,keepaspectratio]{img/h1_ts15.png} &
		\includegraphics[width=0.5\textwidth,keepaspectratio]{img/h2_ts15.png} \\
		(a) Pertinência do conjunto "muito frio" &
		(b) Pertinência do conjunto "frio"
	\end{tabular}
	\caption{\label{imgTS15} Funções de Pertinência.}
\end{figure}

A partir das \jhhref{eqContFuzzy}{equações} são desenvolvidos os controladores para cada uma das regras. A tabela a seguir apresenta os ganhos obtidos:
\begin{center}
	\begin{tabular}{|c|c|}
		\hline
		Regra & Ganho \\ \hline
		 1 & $ K = 
			\begin{bmatrix}
				-13.1962 & 3.0637 & -3.0992 & 0.1430 & 15.0539 & -1.3580\\
				-5.4607 & -15.2912 & 3.4223 & 0.0239 & 1.6964 & 15.9282
			\end{bmatrix}$ \\[20pt] \hline
		2 & $ K = 
			\begin{bmatrix}
				-12.8885 & 1.0745 & -1.5395 & -0.0323 & 14.9563 & -0.5608\\
				-1.7706 & -13.2431 & 1.0214 & -0.0123 & 0.4755 & 15.4671
			\end{bmatrix}$ \\[20pt] \hline
		3 & $ K = 
			\begin{bmatrix}
				-13.1962 & 3.0637 & -3.0992 & 0.1430 & 15.0539 & -1.3580\\
				-5.4607 & -15.2912 & 3.4223 & 0.0239 & 1.6964 & 15.9282
			\end{bmatrix}$ \\[20pt] \hline
		4 & $ K = 
			\begin{bmatrix}
				-12.8885 & 1.0745 & -1.5395 & -0.0323 & 14.9563 & -0.5608\\
				-1.7706 & -13.2431 & 1.0214 & -0.0123 & 0.4755 & 15.4671
			\end{bmatrix}$ \\[20pt] \hline
	\end{tabular}
\end{center}

Os ganhos são sintonizados para o sistema na forma:
\begin{figure}[H]
	\begin{centering}
		\includegraphics[width=\textwidth]{img/modelo_controlado.png}
		\par\end{centering}
	\caption{\label{figPlantCtrl}Espaço de estados da planta controlada}
\end{figure}

Os níveis controlados podem ser observados nas imagens a seguir:
\begin{figure}[H]
	\centering
	\includegraphics[height=0.35\paperheight ,keepaspectratio]{img/ctrl_h1ts2_free.png}
	\caption{\small Nível H1 Controlado }
	\label{figH1TSCtrl2_free}
\end{figure}

\begin{figure}[H]
	\centering
	\includegraphics[height=0.35\paperheight ,keepaspectratio]{img/ctrl_h2ts2_free.png}
	\caption{Nível H2 Controlado }
	\label{figH2CtrlTS2_free}
\end{figure}

No entanto, é importante notar que o controle desenvolvido até aqui não leva em consideração os limites(5 V) dos atuadores (bomba). Incluindo-a ao modelo, tem-se:

Para os sistemas das \jhhref{imgTS5}{imagens} e \ref{imgTS15} são sintonizados de forma similar 15 e 225 ganhos e aplicados ao sistema. As imagens a seguir ilustram os resultados obtidos:
\begin{figure}[H]
	\centering
	\includegraphics[height=0.35\paperheight ,keepaspectratio]{img/ctrl_h1ts2_ulim.png}
	\caption{\small Nível H1 Controlado - Com saturação do Controlador }
	\label{figH1TSCtrl2_ulim}
\end{figure}

\begin{figure}[H]
	\centering
	\includegraphics[height=0.35\paperheight ,keepaspectratio]{img/ctrl_h2ts2_ulim.png}
	\caption{Nível H2 Controlado - Com saturação do Controlador }
	\label{figH2CtrlTS2_ulim}
\end{figure}

Para aliviar a ultrapassagem, dada pelo efeito \textit{windup}, utiliza-se a saturação simples dos integradores nos momentos em que as variáveis manipuladas alcançam seus limites de atuação. Os gráficos a seguir ilustram:
\begin{figure}[H]
	\centering
	\includegraphics[height=0.35\paperheight ,keepaspectratio]{img/ctrl_h1ts2.png}
	\caption{\small Nível H1 Controlado - Com \textit{Anti-Windup}}
	\label{figH1TSCtrl2}
\end{figure}

\begin{figure}[H]
	\centering
	\includegraphics[height=0.35\paperheight ,keepaspectratio]{img/ctrl_h2ts2.png}
	\caption{Nível H2 Controlado - Com \textit{Anti-Windup}}
	\label{figH2CtrlTS2}
\end{figure}

\subsection{Fase Não-Mínima}
Ao contrário da configuração anterior, nesta a maior parte do fluído que saí das bombas é direcionado  para os tanques superiores, ou seja $\gamma_i < 0.5$. A \jhhref{tabFaseNM}{tabela} a seguir apresenta as especificações da planta simulada.

\begin{center} \label{tabFaseNM}
	\begin{tabular}{|c|c|}
		\hline
		\multicolumn{2}{|c|}{Especificações Iniciais da Planta} \\
		\hline
		A1, A3 $(cm^2)$ & 28 \\ \hline
		A2, A4 $(cm^2)$ & 32 \\ \hline
		a1, a3 $(cm^2)$ & 0.071 \\ \hline
		a2, a4 $(cm^2)$ & 0.057 \\ \hline
		g $cm/s$ & 981 \\ \hline
		k1 & 3,15 \\ \hline
		k2 & 3.29 \\ \hline
		$\gamma_1$ & 0.43 \\ \hline
		$\gamma_2$ & 0.34 \\ \hline
		\hline
	\end{tabular}
\end{center}

O \jhhref{eqModNL}{modelo não linear} para esta configuração é dado por:
\begin{equation}
\begin{cases}
\dot{h_{1}} = \frac{1}{A_{1}}(a_{3}\sqrt{2gh_{3}} + \gamma_{1}k_{1}v_{1} - a_{1}\sqrt{2gh_{1}})\\

\dot{h_{2}} = \frac{1}{A_{2}}(a_{4}\sqrt{2gh_{4}} + \gamma_{2}k_{2}v_{2} - a_{2}\sqrt{2gh_{2}})\\

\dot{h_{3}} = \frac{1}{A_{3}}((1 - \gamma_{2})k_{2}v_{2} - a_{3}\sqrt{2gh_{3}})\\

\dot{h_{4}} = \frac{1}{A_{4}}((1 - \gamma_{1})k_{1}v_{1} - a_{4}\sqrt{2gh_{4}})
\end{cases}
\label{eqFNMNL}
\end{equation}

De forma similar, escolhendo os cojuntos fuzzy \{"baixo","alto"\} e definindo \{5 , 15\} como seus representantes os níveis 1 e 2, tem-se os seguintes pontos de linearização:
\begin{center}
	\begin{tabular}{|c|c|c|c|c|c|c|}
		\hline
		Sistema & Nível 1 ($\bar{h_1}$) & Nível 2 ($\bar{h_2}$) & Nível 3 ($\bar{h_3}$) & Nível 1 ($\bar{h_4}$) & Tensão 1 ($\bar{v_1}$) & Tensão 2 ($\bar{v_2}$) \\ \hline
		1 & 5 & 5 & 2.0804 & 1.7175 & 1.8428 & 2.0890 \\ \hline
		2 & 5 & 15 & 0.0321 & 15.9038 & 5.6078 & -0.2595 \\ \hline
		3 & 15 & 5 & 16.9727 & 0.1661 & -0.5730 & 5.9668 \\ \hline
		4 & 15 & 15 & 6.2413 & 5.1525 & 3.1919 & 3.6183 \\	\hline
	\end{tabular}
\end{center}

As imagens a seguir apresenta a comparação entre os modelos \jhhref{eqModNL}{não-linear}, \jhhref{eqModLinear}{linearizado} em um ponto único e \jhhref{eqTakSugPlanta}{Takagi-Sugeno}:

\begin{figure}[H]
	\centering
	\includegraphics[width=\textwidth]{img/h1_ts2_nm.png}
	\caption{\small Nível do Tanque 1}
	\label{figH1TS2_nm}
\end{figure}

\begin{figure}[H]
	\centering
	\includegraphics[width=\textwidth]{img/h2_ts2_nm.png}
	\caption{\small Nível do Tanque 2}
	\label{figH2TS2_nm}
\end{figure}

É notável que o modelo fuzzy representa de modo mais eficiente o sistema. Como dito, o modelo TS pode se aproximar o quanto se desejar do não-linear no qual se baseia. A \jhhref{imgTS5}{imagem} apresenta um modelo com 5 conjuntos para os dois níveis e a \jhhref{imgTS15}{imagem} utilizando 15. É importante notar, no entanto, que a complexidade do modelo é exponencial, devido a combinação dos conjuntos das variáveis linguísticas presentes, assim, o primeiro é composto por 25 ($5^2$) regras Se-Então e o segundo por 225 ($15^2$).

\begin{figure}[H]
	\centering
	\begin{tabular}{cc}
		\includegraphics[width=0.5\textwidth,keepaspectratio]{img/h1_ts5_nm.png} &
		\includegraphics[width=0.5\textwidth,keepaspectratio]{img/h2_ts5_nm.png} \\
		(a) Pertinência do conjunto "muito frio" &
		(b) Pertinência do conjunto "frio"
	\end{tabular}
	\caption{\label{imgTS5_nm} Funções de Pertinência.}
\end{figure}

\begin{figure}[H]
	\centering
	\begin{tabular}{cc}
		\includegraphics[width=0.5\textwidth,keepaspectratio]{img/h1_ts15_nm.png} &
		\includegraphics[width=0.5\textwidth,keepaspectratio]{img/h2_ts15_nm.png} \\
		(a) Pertinência do conjunto "muito frio" &
		(b) Pertinência do conjunto "frio"
	\end{tabular}
	\caption{\label{imgTS15_nm} Funções de Pertinência.}
\end{figure}

A partir das \jhhref{eqContFuzzy}{equações} são desenvolvidos os controladores para cada uma das regras. A tabela a seguir apresenta os ganhos obtidos:
\begin{center}
	\begin{tabular}{|c|c|}
		\hline
		Regra & Ganho \\ \hline
		1 & $ K = 
		\begin{bmatrix}
		-13.1962 & 3.0637 & -3.0992 & 0.1430 & 15.0539 & -1.3580\\
		-5.4607 & -15.2912 & 3.4223 & 0.0239 & 1.6964 & 15.9282
		\end{bmatrix}$ \\[20pt] \hline
		2 & $ K = 
		\begin{bmatrix}
		-12.8885 & 1.0745 & -1.5395 & -0.0323 & 14.9563 & -0.5608\\
		-1.7706 & -13.2431 & 1.0214 & -0.0123 & 0.4755 & 15.4671
		\end{bmatrix}$ \\[20pt] \hline
		3 & $ K = 
		\begin{bmatrix}
		-13.1962 & 3.0637 & -3.0992 & 0.1430 & 15.0539 & -1.3580\\
		-5.4607 & -15.2912 & 3.4223 & 0.0239 & 1.6964 & 15.9282
		\end{bmatrix}$ \\[20pt] \hline
		4 & $ K = 
		\begin{bmatrix}
		-12.8885 & 1.0745 & -1.5395 & -0.0323 & 14.9563 & -0.5608\\
		-1.7706 & -13.2431 & 1.0214 & -0.0123 & 0.4755 & 15.4671
		\end{bmatrix}$ \\[20pt] \hline
	\end{tabular}
\end{center}

Os níveis controlados podem ser observados nas imagens a seguir:
\begin{figure}[H]
	\centering
	\includegraphics[width=\textwidth]{img/nm_ctrl_h1ts2_free.png}
	\caption{\small Nível H1 Controlado }
	\label{figH1TSCtrl2_free_nm}
\end{figure}

\begin{figure}[H]
	\centering
	\includegraphics[width=\textwidth]{img/nm_ctrl_h2ts2_free.png}
	\caption{Nível H2 Controlado }
	\label{figH2CtrlTS2_free_nm}
\end{figure}

No entanto, é importante notar que o controle desenvolvido até aqui não leva em consideração os limites(5 V) dos atuadores (bomba). Incluindo-a ao modelo, tem-se:

Para os sistemas das \jhhref{imgTS5}{imagens} e \ref{imgTS15} são sintonizados de forma similar 15 e 225 ganhos e aplicados ao sistema. As imagens a seguir ilustram os resultados obtidos:
\begin{figure}[H]
	\centering
	\includegraphics[width=\textwidth]{img/nm_ctrl_h1ts2_ulim.png}
	\caption{\small Nível H1 Controlado - Com saturação do Controlador }
	\label{figH1TSCtrl2_ulim_nm}
\end{figure}

\begin{figure}[H]
	\centering
	\includegraphics[width=\textwidth]{img/nm_ctrl_h2ts2_ulim.png}
	\caption{Nível H2 Controlado - Com saturação do Controlador }
	\label{figH2CtrlTS2_ulim_nm}
\end{figure}

Para aliviar a ultrapassagem, dada pelo efeito \textit{windup}, utiliza-se a saturação simples dos integradores nos momentos em que as variáveis manipuladas alcançam seus limites de atuação. Os gráficos a seguir ilustram:
\begin{figure}[H]
	\centering
	\includegraphics[width=\textwidth]{img/nm_ctrl_h1ts2.png}
	\caption{\small Nível H1 Controlado - Com \textit{Anti-Windup}}
	\label{figH1TSCtrl2_nm}
\end{figure}

\begin{figure}[H]
	\centering
	\includegraphics[width=\textwidth]{img/nm_ctrl_h2ts2.png}
	\caption{Nível H2 Controlado - Com \textit{Anti-Windup}}
	\label{figH2CtrlTS2_nm}
\end{figure}
\section{Implementação}
As imagens a seguir apresentam 
\begin{table}[!ht]
	\caption{Ganhos Identificados}
	\label{tabDescPlanta}
	\small
	\centering
	\scalebox{1}{
		\begin{tabular}{|c|c|}
			\hline
			Regra & Ganho \\ \hline
			1 & $ K = 
			\begin{bmatrix}
			1 & 2 & 3 & 4 & 5 & 6 \\
			1 & 2 & 3 & 4 & 5 & 6
			\end{bmatrix}$ \\[20pt] \hline
			2 & $ K = 
			\begin{bmatrix}
			1 & 2 & 3 & 4 & 5 & 6 \\
			1 & 2 & 3 & 4 & 5 & 6
			\end{bmatrix}$ \\[20pt] \hline
			3 & $ K = 
			\begin{bmatrix}
			1 & 2 & 3 & 4 & 5 & 6 \\
			1 & 2 & 3 & 4 & 5 & 6
			\end{bmatrix}$ \\[20pt] \hline
			4 & $ K = 
			\begin{bmatrix}
			1 & 2 & 3 & 4 & 5 & 6 \\
			1 & 2 & 3 & 4 & 5 & 6
			\end{bmatrix}$ \\[20pt] \hline
		\end{tabular}
	}
\end{table}

\begin{figure}[H]
	\centering
	\includegraphics[width=0.7\textwidth]{img/FM_h1_5_10_15.png}
	\caption{Resultado H1}
	\label{figH1TS2}
\end{figure}

\begin{figure}[H]
	\centering
	\includegraphics[width=0.7\textwidth]{img/FM_h1_5_10_15.png}
	\caption{Resultado H2}
	\label{figH1TS2}
\end{figure}

\selectlanguage{brazil}%

